%-------------------------
% Resume in Latex
% Author : Jake Gutierrez
% Based off of: https://github.com/sb2nov/resume
% License : MIT
%------------------------

\documentclass[letterpaper,10.5t]{article}

\usepackage{latexsym}
\usepackage[empty]{fullpage}
\usepackage{titlesec}
\usepackage{marvosym}
\usepackage[usenames,dvipsnames]{color}
\usepackage{verbatim}
\usepackage{enumitem}
\usepackage[hidelinks]{hyperref}
\usepackage{fancyhdr}
\usepackage[english]{babel}
\usepackage{tabularx}
\input{glyphtounicode}


%----------FONT OPTIONS----------
% sans-serif
% \usepackage[sfdefault]{FiraSans}
% \usepackage[sfdefault]{roboto}
% \usepackage[sfdefault]{noto-sans}
% \usepackage[default]{sourcesanspro}

% serif
% \usepackage{CormorantGaramond}
\usepackage{charter}



\pagestyle{fancy}
\fancyhf{} % clear all header and footer fields
\fancyfoot{}
\renewcommand{\headrulewidth}{0pt}
\renewcommand{\footrulewidth}{0pt}

% Adjust margins
\addtolength{\oddsidemargin}{-0.5in}
\addtolength{\evensidemargin}{-0.5in}
\addtolength{\textwidth}{1in}
\addtolength{\topmargin}{-.5in}
\addtolength{\textheight}{1.0in}

\urlstyle{same}

\raggedbottom
\raggedright
\setlength{\tabcolsep}{0in}

% Sections formatting
\titleformat{\section}{
  \vspace{-4pt}\scshape\raggedright\large
}{}{0em}{}[\color{black}\titlerule \vspace{-5pt}]

% Ensure that generate pdf is machine readable/ATS parsable
\pdfgentounicode=1

%-------------------------
% Custom commands
\newcommand{\resumeItem}[1]{
  \item\small{
    {#1 \vspace{-2pt}}
  }
}

\newcommand{\resumeSubheading}[4]{
  \vspace{-2pt}\item
    \begin{tabular*}{0.97\textwidth}[t]{l@{\extracolsep{\fill}}r}
      \textbf{#1} & #2 \\
      \textit{\small#3} & \textit{\small #4} \\
    \end{tabular*}\vspace{-7pt}
}

\newcommand{\resumeSubSubheading}[2]{
    \item
    \begin{tabular*}{0.97\textwidth}{l@{\extracolsep{\fill}}r}
      \textit{\small#1} & \textit{\small #2} \\
    \end{tabular*}\vspace{-7pt}
}

\newcommand{\resumeProjectHeading}[2]{
    \item
    \begin{tabular*}{0.97\textwidth}{l@{\extracolsep{\fill}}r}
      \small#1 & #2 \\
    \end{tabular*}\vspace{-7pt}
}

\newcommand{\resumeSubItem}[1]{\resumeItem{#1}\vspace{-4pt}}

\renewcommand\labelitemii{$\vcenter{\hbox{\tiny$\bullet$}}$}

\newcommand{\resumeSubHeadingListStart}{\begin{itemize}[leftmargin=0.15in, label={}]}
\newcommand{\resumeSubHeadingListEnd}{\end{itemize}}
\newcommand{\resumeItemListStart}{\begin{itemize}}
\newcommand{\resumeItemListEnd}{\end{itemize}\vspace{-5pt}}

%-------------------------------------------
%%%%%%  RESUME STARTS HERE  %%%%%%%%%%%%%%%%%%%%%%%%%%%%


\begin{document}

%----------HEADING----------
% \begin{tabular*}{\textwidth}{l@{\extracolsep{\fill}}r}
%   \textbf{\href{http://sourabhbajaj.com/}{\Large Sourabh Bajaj}} & Email : \href{mailto:sourabh@sourabhbajaj.com}{sourabh@sourabhbajaj.com}\\
%   \href{http://sourabhbajaj.com/}{http://www.sourabhbajaj.com} & Mobile : +1-123-456-7890 \\
% \end{tabular*}

\begin{center}
    \textbf{\Huge \scshape Moncef Bousselat} \\ \vspace{1pt}
    \small Nancy, France $|$ 
    \small +33753542166 $|$ 
    \href{mailto:a.m.bousselat@gmail.com@gmail.com}{\underline{a.m.bousselat@gmail.com}} $|$ 
    \href{https://www.linkedin.com/in/moncefbousselat/}{\underline{linkedin.com/in/moncefbousselat}} $|$
    \href{https://github.com/Somnef}{\underline{github.com/Somnef}} $|$
    \href{https://www.somnef.com}{\underline{somnef.com}}
\end{center}


\section{Formation}
    \resumeSubHeadingListStart
        \resumeSubheading
        {Master en Networking et Cloud Computing}{Sep. 2023 -- Juil. 2025}
        {Université de Lorraine $|$ Leeds Beckett University $|$ Lulea Technical University}{Nancy, FR $|$ Leeds, UK $|$ Skelleftea, SE}
            \resumeItemListStart
                \resumeItem{Sélectionné parmi 2000+ candidats pour la bourse Erasmus Mundus "Green Networking and Cloud Computing"}
                \resumeItem{Modules pertinents: Services Cloud, Data Science, Systèmes Intelligents et Robotique, Internet of Things, Réseaux Sans-fil Avancés, Ingénierie des Systèmes, Réseaux, Qualité de Service et Qualité d'Expérience}
            \resumeItemListEnd
      
        \resumeSubheading
        {Master en Data Science et IA}{Sep. 2018 -- Juil. 2023}
        {E   cole Nationale Polytechnique}{Alger, Algérie}
            \resumeItemListStart
                \resumeItem{Sélectionnés parmi les 10\% des meilleurs étudiants des écoles préparatoires du pays à travers un concours national}
                \resumeItem{Modules pertinents: Bases de Données Avancées, Probabilités et Statistiques, Analyse de Données Multi-variées, Apprentissage Automatique \& Profond, Traitement du Langage Naturel, Business Intelligence, Blockchain, Cloud Computing et Big Data}
            \resumeItemListEnd
  \resumeSubHeadingListEnd


%-----------EXPERIENCE-----------
\section{Expérience Professionnelle}
    \resumeSubHeadingListStart

        \resumeSubheading
        {Stagiaire en Ingénierie Cloud}{Déc 2024 -- Present}
        {Université de Lorraine}{Nancy, France}
            \resumeItemListStart
                \resumeItem{Création d'un système décentralisé d'identité avec Hyperledger Indy pour un partage sécurisé dans le cloud}
                \resumeItem{Déploiement sur AWS avec Docker (ECS) et évaluation des performances du système}
            \resumeItemListEnd

        \resumeSubheading
        {Stagiaire en Développement Full-Stack/Cloud}{Jan 2023 -- Juil 2023}
        {Schlumberger}{Alger, Algérie}
            \resumeItemListStart
                \resumeItem{Création d'une API REST pour enregistrer 1000+ transactions/jour sur une blockchain Hyperledger Fabric}
\resumeItem{Développement full-stack avec Flask, VueJS, Docker et déploiement sur AWS ECS, résolvant un problème critique de pertes d'actifs}
            \resumeItemListEnd

        \resumeSubheading
        {Stagiaire Data et IA}{Oct. 2022 -- Jan. 2023}
        {Ericsson}{Alger, Algeria}
            \resumeItemListStart
                \resumeItem{Mise en oeuvre d'un modèle de reconnaissance d'images utilisant YOLO pour aider les agents de terrain à identifier les appareils, avec une précision de classification de plus de 85\%}
                \resumeItem{Construction et entraînement d'autoencodeur convolutif pour débruiter des images sur PyTorch, atteignant une précision de reconstruction de 99\% sur le dataset MNIST}
        \resumeItemListEnd

        \resumeSubheading
        {Stagiaire Analyste de Données}{Mai 2022 -- Juil. 2022}
        {BH Advisory}{Alger, Algérie}
            \resumeItemListStart
                \resumeItem{Réalisation d'une veille sur le marché des matériaux de construction pour identifier les tendances de prix et de disponibilité}
                \resumeItem{Création de Scrapers pour la collecte de données à partir des sites web des entreprises et présentation sur un dashboard VueJS}
        \resumeItemListEnd

    \resumeSubHeadingListEnd


%-----------PROJECTS-----------
\section{Projets}
    \resumeSubHeadingListStart

        % \resumeProjectHeading
        % {\textbf{Infrastructure AWS de mise à l'échelle automatique déployée avec Terraform} $|$ \emph{AWS, Terraform, Java}}{Jan. 2025}
        %     \resumeItemListStart
        %         \resumeItem{Agents JADE conçus pour attaquer un serveur par DDoS, déployés sur des machines EC2 distribuées capables de générer plus de 20,000 instances en moins de 30 secondes}
        %         \resumeItem{Déploiement d'une infrastructure de serveur évolutive utilisant Terraform pour réagir à l'intensité des attaques en garantissant que l'utilisation du processeur est toujours inférieure à 50\%}
        %         \resumeItem{Création d'une application front-end utilisant VueJS et le SDK AWS pour surveiller les métriques CloudWatch sur les serveurs attaqués}
        %     \resumeItemListEnd

        % \resumeProjectHeading
        % {\textbf{Tolérance aux pannes de systèmes distribués avec Kubernetes} $|$ \emph{Bash, Docker, Kubernetes, Prometheus} $|$ \href{https://github.com/Somnef/kubernetes-fault-tolerance}{\underline{GitHub}}}{Nov. 2024}
        %     \resumeItemListStart
        %         \resumeItem{Conception et déploiement d'une application Python sans état sur des conteneurs Docker orchestrés avec Kubernetes avec mise à l'échelle automatique horizontale des pods (HPA) basée sur l'utilisation du processeur, permettant au système de s'adapter automatiquement à des conditions de charge variables}
        %         \resumeItem{Mise en place d'attaques de test de résistance via un volume élevé de requêtes cURL pour évaluer la scalabilité et les capacités d'auto-réparation du système, démontrant ainsi la capacité de Kubernetes à s'adapter dynamiquement aux fluctuations du trafic et à maintenir une haute disponibilité}
        %         \resumeItem{Utilisation de Prometheus pour surveiller et analyser l'utilisation des ressources, garantissant que le système s'adapte efficacement à l'évolution des modèles de trafic avec un minimum de frais généraux}
        %     \resumeItemListEnd

        % \resumeProjectHeading
        % {\textbf{Apprentissage automatique pour l'évaluation de la qualité de l'eau potable} $|$ \emph{Python, PyTorch}}{Mai 2024}
        %     \resumeItemListStart
        %         \resumeItem{Entraînement de différents modèles d'apprentissage automatique et profond pour évaluer la potabilité des échantillons d'eau en fonction de leurs propriétés chimiques}
        %     \resumeItemListEnd

        % \resumeProjectHeading
        % {\textbf{Profilage de la consommation énergétique du GPU pour l'entraînement des RNA} $|$ \emph{Python, PyTorch, Bash}}{Avr. 2024}
        %     \resumeItemListStart
        %         \resumeItem{Développement d'un outil de profilage GPU, qui exécute des tâches d'entraînement pour plus de 25 combinaisons d'hyperparamètres afin d'explorer l'effet de chacune sur la consommation d'énergie}
        %         \resumeItem{Obtention de la meilleure précision dans le pool de paramètres avec 20\% d'économies d'énergie}
        %     \resumeItemListEnd

        % \resumeProjectHeading
        % {\textbf{Fournisseur de salles basé sur le cloud et l'IoT} $|$ \emph{Python, Arduino, MongoDB, MQTT, AWS}}{Nov. 2023}
        %     \resumeItemListStart
        %         \resumeItem{Déploiement d'une application sur AWS EC2 pour collecter et enregistrer les données de la salle à partir des appareils Arduino (température, bruit, éclairage...) sur MongoDB afin de fournir à l'étudiant la salle qui correspond le mieux à ses critères grâce à un algorithme de prise de décision personnalisé}
        %     \resumeItemListEnd

        % \resumeProjectHeading
        % {\textbf{Business Game} $|$ \emph{VueJS, Laravel, Docker}}{Dec. 2020 - Avr. 2023}
        %     \resumeItemListStart
        %         \resumeItem{Pilotage de l'équipe de développement d'un club étudiant pour construire et améliorer un logiciel de simulation de marché pour deux éditions consécutives de l'événement "Business Game"}
        %         \resumeItem{Optimisation et équilibrage du trafic pour 12 équipes de jeu avec une moyenne de 1 000 requêtes par minute sur les serveurs locaux}
        %     \resumeItemListEnd
          
        \resumeProjectHeading
        {\textbf{Algorithme NEAT appliqué aux jeux vidéos} $|$ \emph{Python, NEAT-Python, Pygame} $|$ \href{https://github.com/Somnef/snake_neat_ai}{\underline{GitHub}}}{Oct. 2022}
            \resumeItemListStart
                \resumeItem{Développement de répliques 1:1 de jeux populaires tels que Flappy Bird et Snake à l'aide de Pygame}
                \resumeItem{Entraînement d'agents avec des algorithmes de neuro-évolution, performants mieux que 100\% des joueurs humains testés}
            \resumeItemListEnd

        \resumeProjectHeading
            {\textbf{Simulateur de feux de forêts 3D} $|$ \emph{Unity, C\#, Python, Google Earth Engine} $|$ \href{https://github.com/Somnef/semi-empirical-wildfire-simulation}{\underline{GitHub}} $|$ \href{https://www.researchgate.net/publication/354678516_Applying_semi-empirical_simulation_of_wildfire_on_real_world_satellite_imagery_data}{\underline{ResearchGate}}}{Dec. 2021}
                \resumeItemListStart
                    \resumeItem{Collecte et segmentation d'images satellites Google avec apprentissage automatique et reconstruction du terrain 3D sur Unity}
                    \resumeItem{Simulation de la propagation d'un feu de forêt sur la scène grâce à un modèle semi-empirique d'automates cellulaires}
                    \resumeItem{Lauréat du concours algérien d'ingénierie}
                \resumeItemListEnd

        \resumeProjectHeading
          {\textbf{Focus AI} $|$ \emph{Python, PyTorch, OpenCV, MediaPipe} $|$ \href{https://github.com/Somnef/focus-monitor-ai}{\underline{GitHub}}}{Nov. 2021}
          \resumeItemListStart
            \resumeItem{Optimisation d'un modèle d'apprentissage profond de suivi du visage pour la surveillance de la concentration, avertissant les utilisateurs lorsqu'une perte de concentration est détectée}
            \resumeItem{Les résultats enregistrés ont montré une augmentation allant jusqu'à 50\% de la concentration et de la productivité des utilisateurs surveillés} 
            \resumeItem{Gagnant du hackathon Google DevFest 2021}
          \resumeItemListEnd

        \resumeProjectHeading
        {\textbf{Scraper de boutique en ligne} $|$ \emph{Python, Selenium} $|$ \href{https://github.com/Somnef/CdiscountScraper}{\underline{GitHub}}}{Avr. 2021}
            \resumeItemListStart
                \resumeItem{Création de scrapers Web pour plusieurs boutiques en ligne (Amazon, CDiscount, Materiel.net) avec enregistrement des prix et comparaison au fil du temps}
                \resumeItem{La comparaison multi-sites a permis de réaliser en moyenne 15\% d'économies sur une sélection de produits}
            \resumeItemListEnd
      
    \resumeSubHeadingListEnd



%
%-----------PROGRAMMING SKILLS-----------
\section{Additionnel}
    \begin{itemize}[leftmargin=0.15in, label={}]
        \small{\item{
        \textbf{Langages de Programmation}{: Python, C/C++, C\#, SQL (MySQL), PHP, JavaScript, HTML/CSS, BASH} \\
        \textbf{Outils Cloud \& DevOps}{: AWS (EC2, S3, VPC, CloudWatch, ECS, CloudFront), Terraform, Docker, Kubernetes (MiniKube), Git, GitHub (GitHub Actions), Prometheus} \\
        \textbf{Outils de Développement}{: VueJS, Node.js, Flask, Laravel, ExpressJS} \\
        \textbf{Outils Data Science \& Machine Learning}{: NumPy, Pandas, Matplotlib, Seaborn, PyTorch, Scikit-learn, Tensorflow, Anaconda } \\
        \textbf{Prix et Certifications}{: Gagnant du Hackathon Arctic Challenge (Suède, 2024), Gagnant de l'Algerian Engineering Competition (Algérie, 2021), Gagnant du Hackathon Google DevFest 21 (Algérie, 2021)}\\
        \textbf{Langues}{: Français (C2), Anglais (C2), Arabe (Langue Maternelle)} \\
        }}
    \end{itemize}


%-------------------------------------------
\end{document}
